
\appendix{Appendix A - Code}

\emph{C Code}
\begin{lstlisting}

  #include <R.h>
  #include <math.h>

  // utilities
  double *create_real_vector(int size){
    double *vec = (double *) Calloc(size,double);
    return vec;
  }

  void destroy_real_vector(double *vector, int size){
    Free(vector);
  }

  double **create_real_matrix(int rows, int cols){
    int i;
    double **mat = (double **) Calloc(rows,double*);
    for(i = 0; i < rows; i++) mat[i] = (double *) Calloc(cols,double);
    return mat;
  }

  double **create_and_copy_real_matrix(int rows, int cols, double *data){
    int i,j;
    double **mat = (double **) Calloc(rows,double*);
    for(i = 0; i < rows; i++){
      mat[i] = (double *) Calloc(cols,double);
      for(j=0;j<cols;++j) mat[i][j] = data[j * rows + i];
    }
    return mat;
  }

  void real_matrix_copy(double **mat, int d1, int d2, double *data){
    int i, j;
    for( i=0; i<d1; ++i)
      for( j=0; j<d2; ++j)
        data[i+d1*j] = mat[i][j];
  }

  void destroy_real_matrix(double **matrix, int rows, int cols){
    int i;
    for(i = 0; i < rows; i++){ Free(matrix[i]); }
    Free(matrix);
  }


  // update of S = a*xx' + b*S
  void chol_up(double **L, double *x, int n, double a, double b, double *work) {
    int i,j;
    double r, c, s;

    memcpy(work,x,sizeof(double)*n);
    for(i=0;i<n;++i) work[i] = sqrt(a/b)*work[i]; // this could be avoided maybe?

    for(i=0;i<n;++i){
      r = sqrt(L[i][i]*L[i][i] + work[i]*work[i]);
      c = r/L[i][i];
      s = work[i] / L[i][i];
      L[i][i] = r;
      for(j=i+1;j<n;++j){
        L[j][i] = (L[j][i] + s*work[j]) / c;
        work[j] = c*work[j] - s*L[j][i];
      }
    }
    for(i=0;i<n;++i){
      for(j=0;j<=i;++j){
        L[i][j] *= sqrt(b);
      }
    }
  }

  void solve(double **L, double *b, double *y, double *x, int n){
    int i,j,k;

    y[0] = b[0] / L[0][0];
    for( i=0; i<n; ++i ){
      y[i] = b[i];
      for(j=0; j<i; ++j) y[i] -= y[j] * L[i][j];
      y[i] = y[i] / L[i][i];
    }
    x[n-1] = y[n-1] / L[n-1][n-1];
    for( i=(n-2); i>-1; --i ){
      x[i] = y[i];
      for(j=(n-1); j>i; --j) x[i] -= x[j] * L[j][i];
      x[i] = x[i] / L[i][i];
    }
  }

  // n-variate DCC Model Filter
  void dcc_filter(int *status, double *_L, double *_eps, double *loglik, double *param, double *shrinkage_coefs, double *_shrinkage_vectors, int *T, int *N, int *n){

    int i,j,t;
    double logden;
    double alpha, beta, L_tilde_denom;
    double *y, *x;
    double *work1;
    double **L, **eps, **L_tilde, **shrinkage_vectors;

    *loglik = 0;

    // sanity check
    if( !finite(param[0]) || !finite(param[1]) ){
      *loglik = HUGE_VAL;
      return;
    }

    alpha = param[0];
    beta  = param[1];

    // check constraints
    if( alpha <= 1e-5 || beta < 0 || (alpha+beta)>1 ){
      *loglik = HUGE_VAL;
      return;
    }

    // allocate
    work1   = create_real_vector(*N);
    x       = create_real_vector(*N);
    y       = create_real_vector(*N);
    L       = create_and_copy_real_matrix(*N,*N, _L);
    L_tilde = create_and_copy_real_matrix(*N,*N, _L);
    eps     = create_and_copy_real_matrix(*T, *N, _eps);
    shrinkage_vectors = create_and_copy_real_matrix(*n, *N, _shrinkage_vectors);

    // First part of log-likelihood: log determinant of Rt
    logden        = 0;
    L_tilde_denom = 0;
    // Calculate Q^(-1/2) * L
    for(i = 0; i < *N; ++i){
      for(j = 0; j <= i; ++j) L_tilde_denom += L[i][j] * L[i][j];
      for(j = 0; j <= i; ++j) L_tilde[i][j]  = L[i][j] / sqrt(L_tilde_denom);
      L_tilde_denom = 0;
    }

    // Add log determinant
    // Add likelihoods from each time step

    for(t=1; t<*T; ++t ){

      // Update L
      // update of S = a*S + b*x*x'
      chol_up(L, eps[t-1], *N, alpha, beta, work1);

      for(i=0;i<*n;++i){
        chol_up(L, shrinkage_vectors[i], *N, (1 - alpha - beta) * shrinkage_coefs[i], 1, work1);
      }

      // Calculate Q^(-1/2) * L
      for(i = 0; i < *N; ++i){
        L_tilde_denom = 0;
        for(j = 0; j <= i; ++j) L_tilde_denom += L[i][j] * L[i][j];
        for(j = 0; j <= i; ++j) L_tilde[i][j] = L[i][j] / sqrt(L_tilde_denom);
      }

      for(i = 0; i < *N; ++i) *loglik += log(L_tilde[i][i]);

      //if( finite(logden) ){
      //  *loglik += logden;
      //}
      //else{
      //  Rprintf("problem at time %d\n",t);
      //}

      solve(L_tilde, eps[t], y, x, *N);

      for(i = 0; i < *N; ++i) *loglik += x[i] * eps[t][i];


    }
    // safeguard
    if( !isfinite(*loglik) ){
      *loglik = HUGE_VAL;
    }

    // cleanup
    destroy_real_matrix(L_tilde,*N,*N);
    destroy_real_vector(y,*N);
    destroy_real_vector(x,*N);
    destroy_real_matrix(L,*N, *N);
    destroy_real_matrix(shrinkage_vectors,2, *N);
    destroy_real_matrix(eps,*T, *N);
    destroy_real_vector(work1,*N);
  }
  //
\end{lstlisting}

\emph{R Code}
\lstset{language=R}
\begin{lstlisting}
  dyn.load('/home/euan/documents/BGSE_masters_thesis/R-Package-dynamo/src/dynamo.so')
  source('/home/euan/documents/BGSE_masters_thesis/src/R/data_prep.R')

  linear_shrinkage <- function(eps, n, rho){
    # Linear shrinkage estimator of covariance function
    # eps: matrix of de-garched returns
    # n: number of eigenvalues to take
    T             <- ncol(eps)
    N             <- nrow(eps)

    sample_covariance_matrix  <- (1/T)*(eps %*% t(eps))
    eigenvals                 <- eigen(sample_covariance_matrix)

    eigenvectors  <- eigenvals$vectors
    lambda        <- eigenvals$values

    lambda_bar    <- mean(lambda)
    C_bar         <- matrix(rep(0,(N**2)), N,N)
    shrink_coef   <- rep(0,n)
    for (i in 1:n){
      shrink_coef[i]  <- rho * lambda_bar + (1 - rho)*lambda[i]
      C_bar           <- C_bar + shrink_coef[i] * (eigenvectors[,i] %*% t(eigenvectors[,i]))
    }
    return(list(coefficients = shrink_coef, vectors = eigenvectors[,1:n]))
  }

  dcc.filter <- function( shrinkage_coefs, shrinkage_vectors, eps , params , L){

    n   <- nrow(shrinkage_vectors)
    T   <- nrow(eps)
    N   <- ncol(eps)

    if( any(!is.finite(params)) ){
      filter = list( loglik=Inf , sigma2=rep(NA,T) )
      return( filter )
    }

    result <- .C( 'dcc_filter',
                  status = as.integer(0),
                  L = as.double(L) ,
                  eps = as.double(eps) ,
                  loglik = as.double(0) ,
                  params = as.double(params) ,
                  as.double(shrinkage_coefs),
                  as.double(shrinkage_vectors),
                  as.integer(T),
                  as.integer(N),
                  as.integer(n),
                  PACKAGE="dynamo"
    )

    return(list('loglik' = result$loglik))
    #$
  }

  llh <- function(params){
    return(as.matrix(dcc.filter(coefs, vectors, t(eps), params, t(L))$loglik))
  }
  #$
  DCC.fit <- function(X, n, rho, Trace = 3){
    # Given matrix of returns, fits DCC model

    # Step 1: degarching the data
    N <- ncol(X)
    T <- nrow(X)
    eps <- matrix(rep(0, N*T), N,T)
    for(i in 1:N){
      eps[i,] <- X[,i]/sqrt(garchFit(data = X[,i], trace = F)@h.t)
    }

    # Step 2: Optimising the DCC model parameters
    omega  <- (1/T)*(eps %*% t(eps))
    L   <- t(chol(omega))

    shrink_est  <- linear_shrinkage(eps, n, rho)

    coefs <- shrink_est$coefficients; vectors = shrink_est$vectors

    llh <- function(params){
      return(as.matrix(dcc.filter(coefs, t(vectors), t(eps), params, L)$loglik))
    }
    #$
    params <- c(0.05,0.9)
    lowerBounds <- c(10E-6, 10E-6)
    upperBounds <- 1 - lowerBounds
    if(Trace > 0){fit = nlminb(start = params, objective = llh,
                             lower = lowerBounds, upper = upperBounds, control =  list(trace=Trace))}
    else{  fit = nlminb(start = params, objective = llh,
                        lower = lowerBounds, upper = upperBounds)}
    return(list(par = fit$par, loglik = -0.5*fit$objective))

  }

\end{lstlisting}
