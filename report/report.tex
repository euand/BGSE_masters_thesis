%%%%%%%%%%%%%%%%%%%%%%%%%%%%%%%%%%%%%%%%%
% Short Sectioned Assignment
% LaTeX Template
% Version 1.0 (5/5/12)
%
% This template has been downloaded from:
% http://www.LaTeXTemplates.com
%
% Original author:
% Frits Wenneker (http://www.howtotex.com)
%
% License:
% CC BY-NC-SA 3.0 (http://creativecommons.org/licenses/by-nc-sa/3.0/)
%
%%%%%%%%%%%%%%%%%%%%%%%%%%%%%%%%%%%%%%%%%

%----------------------------------------------------------------------------------------
%	PACKAGES AND OTHER DOCUMENT CONFIGURATIONS
%----------------------------------------------------------------------------------------

\documentclass{article} % A4 paper and 11pt font size

\usepackage[T1]{fontenc} % Use 8-bit encoding that has 256 glyphs
%\usepackage{fourier} % Use the Adobe Utopia font for the document - comment this line to return to the LaTeX default
\usepackage[english]{babel} % English language/hyphenation
\usepackage{amsmath,amsfonts,amsthm} % Math packages

\usepackage{graphicx}
\usepackage{pdfpages}
%\usepackage{hyperref}

\usepackage{sectsty} % Allows customizing section commands
\allsectionsfont{\normalfont} % Make all sections centered, the default font and small caps
\usepackage{tikz}
\usetikzlibrary{arrows}

%\usepackage{titlesec}
%\titleformat*{\subsubsection}{\bfseries}

\usepackage[margin=1in]{geometry}

\usepackage{fancyhdr} % Custom headers and footers
\pagestyle{fancyplain} % Makes all pages in the document conform to the custom headers and footers
\fancyhead{} % No page header - if you want one, create it in the same way as the footers below
\fancyfoot[L]{} % Empty left footer
\fancyfoot[C]{} % Empty center footer
\fancyfoot[R]{\thepage} % Page numbering for right footer
\renewcommand{\headrulewidth}{0pt} % Remove header underlines
\renewcommand{\footrulewidth}{0pt} % Remove footer underlines
\setlength{\headheight}{12pt} % Customize the height of the header
\setlength{\parskip}{1em}

\numberwithin{equation}{section} % Number equations within sections (i.e. 1.1, 1.2, 2.1, 2.2 instead of 1, 2, 3, 4)
\numberwithin{figure}{section} % Number figures within sections (i.e. 1.1, 1.2, 2.1, 2.2 instead of 1, 2, 3, 4)
\numberwithin{table}{section} % Number tables within sections (i.e. 1.1, 1.2, 2.1, 2.2 instead of 1, 2, 3, 4)

\setlength\parindent{0pt} % Removes all indentation from paragraphs - comment this line for an assignment with lots of text

%----------------------------------------------------------------------------------------
%	TITLE SECTION
%----------------------------------------------------------------------------------------

\newcommand{\horrule}[1]{\rule{\linewidth}{#1}} % Create horizontal rule command with 1 argument of height

\title{
\normalfont \normalsize
\textsc{Barcelona Graduate School of Economics} \\ [20pt] % Your university, school and/or department name(s)
\horrule{0.5pt} \\[0.1cm] % Thin top horizontal rule
\Large Efficient Estimation of Dynamic Conditional Correlation Models \\ % The assignment title
\horrule{0.5pt} \\[0.1cm] % Thick bottom horizontal rule
}

\author{Euan Dowers} % Your name

\date{\normalsize\today} % Today's date or a custom date

\begin{document}

\maketitle % Print the title

\tableofcontents

\pagebreak

%----------------------------------------------------------------------------------------
% NEW SECTION
%----------------------------------------------------------------------------------------

\section{Introduction}

The aim of this paper is to allow the DCC model to be efficiently estimated in large dimensions, using two innovations. The first, due to Engle, Ledoit and Wolf \cite{engle ledoit and wolf}, is to use the linear shrinkage method of Ledoit and Wolf \cite{ledoit and wolf} to estimate the correlation targeting matrix of the DCC model, and the second, which is the original innovation of this paper, is to calculate the log-likelihood of the DCC model using a series of rank-one updates to the cholesky decomposition of the quasi-correlation matrix at each time period, thus avoiding having to refactor this matrix at every time period.

The structure of this report is as follows: Section \ref{section: dcc} will describe the Dynamic Conditional Correlation model, including the estimation problem; Section \ref{section: shrinkage} will describe the linear shrinkage estimation for covariance matrices of Ledoit and Wolf.

%----------------------------------------------------------------------------------------
% NEW SECTION
%----------------------------------------------------------------------------------------

\section{Dynamic Conditional Correlation Model} \label{section: dcc}

In order to describe the Dynamic Conditional Correlation Model, it is necessary to build up some notation.
\begin{enumerate}
  \item Let $r_{i,t}$ denote the return of asset $i$ at time $t$, and $r_t$ be the $N$-dimensional vector of all returns at time $t$.
  \item Let $d_{i,t}^2 = V_{t-1}(r_{i,t})$ be the conditional variance of asset $i$ at time $t$, given information up to time $t-1$.
  \item Let $D_t$ be a diagonal matrix whose i-th element is $d_{it}$.
  \item Let $H_t$ be the conditional covariance matrix whose $i,j$-th element is the conditional covariance between assets $i$ and $j$ at time $t$.
  \item Let $\epsilon_{i,t} = \frac{r_{i,t}}{d_{i,t}}$ be the standardised residuals of asset $i$ at time $t$.
  \item Let $R_t$ be the conditional correlation matrix, whose i,j-th entry is given by the conditional correlation between asset $i$ and asset $j$ at time $t$.
  \item Let $\sigma_i^2$ be the unconditional variance of the series $r_i,t$
  \item Let $R$ be the unconditional correlation matrix of the system.
\end{enumerate}

%----------------------------------------------------------------------------------------

\subsection{Dynamic Conditional Correlation Model}

The Dynamic Conditional Correlation model of Engle \cite{engle2002} is as follows. Assume we have $N$ financial assets, and we observe for each of these assets a series of returns over a time period $1, \ldots , T$.

Since the conditional correlation and conditional covariance are related by the equation
\begin{equation}
  \rho_{i,j,t} = \frac{E_{t-1}((r_{i,t} - E_{t-1}(r_{i,t}))(r_{j,t} - E_{t-1}(r_{j,t})))}
                      {\sqrt{V_{t-1}(r_{i,t})V_{t-1}(y_{j,t})}},
\end{equation}
these entries are therefore also given by
\begin{equation}
  \frac{H_{i,j,t}}{\sqrt{H_{i,i,t}H_{j,j,t}}}.
\end{equation}
Therefore, the conditional correlation matrix and conditional variance matrix are given by
\begin{equation}
  R_t = D_t^{-1} H_t D_t ^{-1} \ \  D_t ^2 = diag[H_t]
\end{equation}

To model the quantities $d_{i,t}$ we apply the univariate GARCH(1,1) model to each asset serparately to obtain the dynamics of its conditional variance.
\begin{equation}
  H_{i,i,t} = \omega_i + \alpha_i r_{t-1}^2 + \beta_i H_{i,i,t-1}.
\end{equation}
 Then, the standardised residuals $\epsilon _{i,t}$ are calculated as described above.

In the mean-reverting DCC model that we use, the dynamics of the quasi-correlation matrix $Q_t$ are governed by the process
\begin{equation}
  Q_t = \Omega + \alpha \epsilon_{t-1}\epsilon_{t-1}^T + \beta Q_{t-1}.
\end{equation}
In the correlation targeting version of this model, the intercept matrix $\Omega$ is given
\begin{equation}
  \Omega = (1 - \alpha - \beta)R
\end{equation}

%----------------------------------------------------------------------------------------
% NEW SECTION
%----------------------------------------------------------------------------------------

\section{Shrinkage Estimation}\label{section: shrinkage}

\section{DCC Model Estimation}\label{section: shrinkage}

\section{DCC Model Estimation}\label{section: shrinkage}

\begin{thebibliography}{9}
\bibitem{engle2002}
Engle, R.
\textit{Dynamic Conditional Correlation - a simple class of multivariate models}.
Journal of Business and Economic Statistics, 2002.

\bibitem{bollerslev90}
Bollerslev, D
\textit{Dynamic Conditional Correlation - a simple class of multivariate models}.
Journal of Business and Economic Statistics, 1990.

\bibitem{anticipating correlations}
Engle, R.
\textit{Anticipating Correlations: a new paradigm for risk management}
\end{thebibliography}

\end{document}
